\documentclass[a4paper,10pt]{article}

%A Few Useful Packages
\usepackage{marvosym}
\usepackage{fontspec} 					%for loading fonts
\usepackage{xunicode,xltxtra,url,parskip} 	%other packages for formatting
\RequirePackage{color,graphicx}
\usepackage[usenames,dvipsnames]{xcolor}
\usepackage[big]{layaureo} 				%better formatting of the A4 page
% an alternative to Layaureo can be ** \usepackage{fullpage} **
\usepackage{supertabular} 				%for Grades
\usepackage{titlesec}					%custom \section

%Setup hyperref package, and colours for links
\usepackage{hyperref}
\definecolor{linkcolour}{rgb}{0,0.2,0.6}
\hypersetup{colorlinks,breaklinks,urlcolor=linkcolour, linkcolor=linkcolour}

%FONTS
\defaultfontfeatures{Mapping=tex-text}
%\setmainfont[SmallCapsFont = Fontin SmallCaps]{Fontin}
%%% modified for Karol Kozioł for ShareLaTeX use
\setmainfont[
  SmallCapsFont = fonts/Fontin-SmallCaps.otf,
  BoldFont = fonts/Fontin-Bold.otf,
  ItalicFont = fonts/Fontin-Italic.otf
  ]
  {Fontin.otf}
  %%%

  %CV Sections inspired by:
  %http://stefano.italians.nl/archives/26
  \titleformat{\section}{\Large\scshape\raggedright}{}{0em}{}[\titlerule]
  \titlespacing{\section}{0pt}{3pt}{3pt}
  %Tweak a bit the top margin
  %\addtolength{\voffset}{-1.3cm}

  %Italian hyphenation for the word: ''corporations''
  \hyphenation{im-pre-se}

  %-------------WATERMARK TEST [**not part of a CV**]---------------
  \usepackage[absolute]{textpos}

  \setlength{\TPHorizModule}{30mm}
  \setlength{\TPVertModule}{\TPHorizModule}
  \textblockorigin{2mm}{0.65\paperheight}
  \setlength{\parindent}{0pt}

  %--------------------BEGIN DOCUMENT----------------------
  \begin{document}

  %WATERMARK TEST [**not part of a CV**]---------------
  %\font\wm=''Baskerville:color=787878'' at 8pt
  %\font\wmweb=''Baskerville:color=FF1493'' at 8pt
  %{\wm
  %	\begin{textblock}{1}(0,0)
  %		\rotatebox{-90}{\parbox{500mm}{
  %			Typeset by Alessandro Plasmati with \XeTeX\  \today\ for
  %			{\wmweb \href{http://www.aleplasmati.comuv.com}{aleplasmati.comuv.com}}
  %		}
  %	}
  %	\end{textblock}
  %}

  \pagestyle{empty} % non-numbered pages

  \font\fb=''[cmr10]'' %for use with \LaTeX command

  %--------------------TITLE-------------
  \par{\centering
  {\Huge André Luís \Huge{Carvalho} Moreira
  }\bigskip\par}

  %--------------------SECTIONS-----------------------------------
  %Section: Personal Data
  \section{Personal Data}

  \begin{tabular}{rl}
    %\textsc{Place and Date of Birth:} & Someplace, Italy  | dd Month 1912 \\
    \textsc{Location:}   & Fortaleza, Ceará, Brazil \\
    \textsc{Mobile:}     & +55 85 992-703-813\\
    \textsc{email:}     & {andrelcmoreira@disroot.org}
  \end{tabular} \\

  %Section: Work Experience 1
  \section{Work Experience}
  \begin{tabular}{r|p{11cm}}

    \textsc{Oct 2020 - Current} &\textbf{Senior System Analyst at PagSeguro via Invillia - Araraquara, Brazil} \\&\footnotesize{Works as senior system analyst, leading a team responsible for developing embedded software for a family of POS (Point of sale) devices running Android, as well as developing native libraries and maintaining legacy software for the same platforms.}
    \\ & \footnotesize{\textbf{Keywords:} C/C++, CMake, Android, GTest/GMock, JNI, POS.}
    \\\multicolumn{2}{c}{} \\

    \textsc{Mar 2020 - Mar 2021} &\textbf{Software Development Analyst at HPE via Instituto Atlântico - Fortaleza, Brazil} \\&\footnotesize{Integrated the R\&D team responsible for the development of security solutions targeting cloud environments. The developed solutions provides an efficient way to protect virtualized and bare-metal machines running unix-like operating systems against tampering, using techniques like VM introspection and kernel data structures analyzing.}
    \\ & \footnotesize{\textbf{Keywords:} C, Python, Linux, Kernel programming, VM introspection, KUnit, Robot framework, Security.}
    \\\multicolumn{2}{c}{} \\

    \textsc{Sep 2018 - Mar 2020} &\textbf{Software Developer at Datacom via LDS/IFCE - Fortaleza, Brazil} \\&\emph{Software Development Laboratory of Federal Institute of Education, Science
    and Technology of Ceará}\\&\footnotesize{Worked as embedded software developer, focusing on the development of applications and technologies targeting the computer networks context. Integrated the platform team, being responsible for developing device drivers - for SFP and PSU devices - targeting the PowerPC and X86 architectures. Furthermore,  made researches regarding white-box switches and network operating systems (NOS) as well.
    Also, integrated the services team, being responsible for developing CLI applications using YANG/ConfD, C/C++, Python and Shell script.}
    \\ & \footnotesize{\textbf{Keywords:} C/C++, Python, Shell script, ConfD,  GTest/GMock, Robot framework, Device tree, Device drivers, Linux, Buildroot.}
    \\\multicolumn{2}{c}{} \\

    \textsc{Dec 2017 - Sep 2018} & \textbf{Software Developer at Lenovo via LSBD/UFC - Fortaleza, Brazil} \\&\emph{Systems and Database Laboratory of Federal University of Ceará}\\&\footnotesize{Integrated the development team responsible for developing/maintaining a multiplatform software (worldwide used) for hardware diagnostics using C++ and Qt, and developing a multiplatform/multithreading/multiprocessing API responsible for providing the basic diagnostic functionalities for this software, using modern C++.}
    \\ & \footnotesize{\textbf{Keywords:} C++14, Qt, GTest/GMock, CMake, Linux.}
    \\\multicolumn{2}{c}{} \\

    \textsc{Jan 2016 - Dec 2017} & \textbf{Software Developer at Datacom via LDS/IFCE - Fortaleza, Brazil} \\&\emph{Software Development Laboratory of Federal Institute of Education, Science
    and Technology of Ceará}\\&\footnotesize{Worked as embedded software developer, focusing on the development of computing network applications and protocols. Integrated the team responsible for implementing L3/L4 protocols, such as TWAMP (based on an open-source implementation of OWAMP) and VRRP (written from scratch), both developed based on technical specification (RFC) and using modern C++.}
    \\ & \footnotesize{\textbf{Keywords:} C/C++11, GTest/GMock, CMake, TCP/IP stack, Networking protocols, Linux.}
    \\\multicolumn{2}{c}{} \\
  \end{tabular}

  \begin{tabular}{r|p{11cm}}
    \textsc{Aug 2014 - Jan 2016} & \textbf{Intern at IOCT - Fortaleza, Brazil}\\&\emph{Orion Institute of Science and Technology}\\&\footnotesize{Worked with embedded software on projects involving real-time audio and video recording/streaming and real-time audio acquisition/decoding using Freescale i.MX53 and i.MX6 based boards, respectively. I integrated the team responsible for developing the frontend/backend of the embedded application, using C/C++ and Qt, and generating/maintaining the operating system images (based on linux) used by both the targets, using YOCTO.}
    \\ & \footnotesize{\textbf{Keywords:} C/C++, Qt, Yocto, Device tree, Device drivers, Linux.}
    \\\multicolumn{2}{c}{} \\

    \textsc{Sep 2013 - Aug 2014} & \textbf{Undergraduate Researcher at GCEM - Fortaleza, Brazil} \\&\emph{Group of research in embedded computing at Federal Institute of Education,
    Science and Technology of Ceara, IFCE}\\&\footnotesize{Worked with R\&D. I was responsible for developing a PoC of a modular Indoor positioning system based on ultrassound and infrared communication. Such system aimed to be used as part of an autonomous robot. The prototypes were developed in a PIC18-based board, using C.}
    \\ & \footnotesize{\textbf{Keywords:} C, PIC18, PCB Design, Altium designer, Digital electronics.}
    \\\multicolumn{2}{c}{}

  \end{tabular}

  %Section:  Formal Education
  \section{Education}
  \begin{tabular}{rl}
    \textsc{Sep} 2013 - Interrupted & Bachelor of Computer Engineering \\ & \textbf{Federal Institute of Education, Science and Technology of Ceara, IFCE}, \\ & Fortaleza,  Brazil
  \end{tabular} \\

  %Section: Projects
  \section{Projects}
  \begin{tabular}{rl}
    \textbf{Void Linux:}& General purpose operating system based on Linux kernel. (Contributor)\\
    \textbf{doxycheck:}& Test doxygen code comments using doxygen/sphinx+breathe. (Contributor)\\
      \textbf{usb-rofi:}& Tool to manage USB flash drives using rofi and udev. (Author)\\
      \textbf{pool-day:}& Thread pool library. (Author)\\
      \textbf{rsh:}& Simple and portable reverse shell. (Author)\\
    \end{tabular} \\

    %Section: Complementary Education
    \section{Complementary Education}
    \begin{tabular}{rl}
      \textsc Dec 2021 - Dec 2021 & Building debian packages\\ Institution & Kretcheu  \\ \textsc Credit hours & 30 hours \\&\\
      \textsc Aug 2020 - Aug 2020 & Vagrant for beginners \\ Institution & Udemy  \\ \textsc Credit hours & 1,5 hours \\&\\
      \textsc Jun 2020 - Jun 2020 & Docker: Essential tool for developers \\ Institution & Udemy  \\ \textsc Credit hours & 5,5 hours \\&\\
      \textsc Jun 2011 - Jul 2011 & Altium designer \\ Institution & University of Fortaleza  \\ \textsc Credit hours & 40 hours \\&\\
      \textsc Aug 2011 - Aug 2011 & PIC18 microcontrollers - C Programming \\ Institution & University of Fortaleza \\ \textsc Credit hours & 40 hours \\
    \end{tabular} \\

    %Section: Languages
    \section{Languages}
    \begin{tabular}{rl}
      \textsc{Portuguese:}& Mothertongue\\
      \textsc{English:}& Intermediate\\
      \textsc{Spanish:}& Basic\\
    \end{tabular}


    %Section: Skilss
    \section{Skills}
    \begin{center}
      \begin{tabular}{| l | l | l | l |}
        \hline
        Operating Systems & Linux-like and Windows\\ \hline
        Programming Languages & C, C++ (11 and 14), Python and Shell script\\ \hline
        Debuggers & GDB \\ \hline
        Frameworks & Qt, GTest/GMock, KUnit, Flask and Robot\\ \hline
        Version Control & Git\\ \hline
        Build Tools & Make, CMake and Buildroot\\ \hline
        Continuous Integration & Jenkins and Github Actions\\ \hline
        Bugtrackers & Jira and Trello\\ \hline
        Databases & MongoDb\\ \hline
        Virtualization & Docker, LXC, Vagrant and Virtualbox\\ \hline
      \end{tabular}
    \end{center}

    \section{Links}
    \begin{tabular}{rl}
      \textsc{Github:}& https://github.com/carvalhudo\\
      \textsc{Linkedin:}&https://www.linkedin.com/in/andré-carvalho-08202682 \\
    \end{tabular}

  \end{document}
